\appendix
\section{Resources}

The example code from this presentation is available on my Github (\url{http://github.com/hoelzro/xmpp-talk}), along
with the \LaTeX\ code used for the presentation itself.

\subsection*{Websites}

\begin{description}
\item[\url{http://xmpp.org/services}] A list of publicly available XMPP servers to use for chat.
\item[\url{http://xmpp.org/extensions}] A list of published XMPP extensions.
\item[\url{http://xmpp.org/rfcs}] Lists the RFCs that define XMPP.
\item[\url{http://xmpp.org/xmpp-software/servers}] A list of XMPP server implementations.
\item[\url{http://xmpp.org/xmpp-software/clients}] A list of XMPP clients.
\item[\url{http://xmpp.org/xmpp-software/libraries}] A list of XMPP libraries.
\item[\url{http://prosody.im/}] Homepage of the Prosody XMPP server.
\item[\url{http://danga.com/djabberd/}] Homepage of the djabberd XMPP server.
\item[\url{http://www.ejabberd.im/}] Homepage of the ejabberd XMPP server.
\end{description}

\subsection*{Books}

\begin{description}
\item[XMPP: The Definitive Guide (ISBN: 978-0-596-52126-4)] This book is a fantastic and thorough introduction to XMPP.
\item[Professional XMPP (ISBN: 978-0-470-54071-8)] This book offers a quick introduction to XMPP, but focuses mainly on developing browser-based XMPP applications.
\end{description}

\subsection*{XMPP Services of Note}

\begin{description}
\item[Chesspark] Chesspark is a chess game service that uses XMPP to communicate players' moves.
\item[Collecta]  Collecta is a real-time search engine that delivers streaming results using XMPP.
\item[Identica]  Identica is a microblogging service that uses XMPP to deliver updates to its users.
\end{description}
