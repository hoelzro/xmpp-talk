\part{Writing an XMPP Bot with AnyEvent::XMPP}

\newpage
Now we're going to go through how you would develop an XMPP bot
using AnyEvent::XMPP.  We'll start simple, but by the time we're
finished, we'll have ourselves a bot capable of handling both one-on-one
and multiuser conversations.

\pause
So, without further ado, here is iteration 1 of the bot:

\pause

\begin{shaded}
\inputminted{perl}{examples/bare-bot.pl}
\end{shaded}

\pause

Now that we've got some code, let's walk through the example...

\newpage

\begin{shaded}
\begin{minted}{perl}
#!/usr/bin/env perl

# comments...

use strict;
use warnings;
\end{minted}
\end{shaded}

\pause
\begin{comment}
You should be familiar with this part already!
\end{comment}

\begin{shaded}
\begin{minted}{perl}
use AnyEvent::XMPP::IM::Connection;
\end{minted}
\end{shaded}

\pause
\begin{comment}
This statement loads the AnyEvent::XMPP::IM::Connection class.  Normally I wouldn't
waste my time explaining a use statement, but I just want to make sure you know there
are three connection classes in AnyEvent::XMPP:

\begin{description}
\item[AnyEvent::XMPP::Connection] Simple connections; no bells and whistles.
\item[AnyEvent::XMPP::IM::Connection] Provides more advanced events (message, contact_request_subscribe), initial presence, roster.
\item[AnyEvent::XMPP::Client] Manages multiple connections.
\end{description}
\end{comment}

\begin{shaded}
\begin{minted}{perl}
my $cond = AnyEvent->condvar;

# some code in the middle...

$cond->recv;
\end{minted}
\end{shaded}

\begin{comment}
This is how you run an AnyEvent event loop.  If you're not familiar with AnyEvent, I suggest you read up on it; however, intimate knowledge of it is not necessary for this tutorial.
\end{comment}

\newpage
\begin{shaded}
\begin{minted}{perl}
my $conn = AnyEvent::XMPP::IM::Connection->new(
    jid      => 'rob@localhost',
    password => 'abc123',
);

$conn->connect;
\end{minted}
\end{shaded}

\begin{comment}
Creates a connection object and connects when the event loop starts running.
Authenticates as rob@localhost, using the password 'abc123'.
\end{comment}
