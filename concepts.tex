\part{Concepts}

\newpage
\begin{center}
{\huge \bfseries JIDs}
\end{center}
\pause

JID is an acronym for Jabber ID.  A JID uniquely identifies a user on an XMPP network.

\pause

Here's an example JID (that happens to be mine): \jid{hoelzro@talkr.im}

\pause

What I didn't tell you before is that there are actually two types of JIDs!  There are full JIDs
and bare JIDs.  \jid{hoelzro@talkr.im} is a bare JID.  An example of a full JID would be \jid{hoelzro@talkr.im/Home}.
A bare JID uniquely identifies a user on a network, but a full JID uniquely identifies a connection on that
network.

\pause

What does this mean?  This means that a single user can have multiple connections to the same network.  So I can be
connected at home (\jid{hoelzro@talkr.im/Home}), at work (\jid{hoelzro@talkr.im/Work}), and on my phone (\jid{hoelzro@talkr.im/Droid}).

\pause

A full JID has three parts:
\pause

\begin{itemize}
\item The username (hoelzro)
\pause
\item The domain (talkr.im)
\pause
\item The resource (Home)
\end{itemize}
\pause

Bare JIDs have a username and a domain, but no resource.

\pause

Note that although the domain tends to be the server you're connecting to, this is not always the case!  For example, I have
a test server at home that manages the domain "chat.eridanus", but I can connect to it as localhost.  More on this later.
