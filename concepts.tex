\part{Concepts}

\newpage
\section*{JIDs}
\pause

JID is an acronym for Jabber ID.  A JID uniquely identifies a user on an XMPP network.

\pause

Here's an example JID (that happens to be mine): \jid{hoelzro@talkr.im}

\pause

What I didn't tell you before is that there are actually two types of JIDs!  There are full JIDs
and bare JIDs.  \jid{hoelzro@talkr.im} is a bare JID.  An example of a full JID would be \jid{hoelzro@talkr.im/Home}.
A bare JID uniquely identifies a user on a network, but a full JID uniquely identifies a connection on that
network.

\pause

What does this mean?  This means that a single user can have multiple connections to the same network.  So I can be
connected at home (\jid{hoelzro@talkr.im/Home}), at work (\jid{hoelzro@talkr.im/Work}), and on my phone (\jid{hoelzro@talkr.im/Droid}).

\pause

A full JID has three parts:
\pause

\begin{itemize}
\item The username (hoelzro)
\pause
\item The domain (talkr.im)
\pause
\item The resource (Home)
\end{itemize}
\pause

Bare JIDs have a username and a domain, but no resource.

\pause

Note that although the domain tends to be the server you're connecting to, this is not always the case!  For example, I have
a test server at home that manages the domain "chat.eridanus", but I can connect to it as localhost.  More on this later.

\newpage
\section*{Stanzas}
\pause

Stanzas are the basic form of communication in XMPP.  There are three kinds:

\pause
\subsection*{Presence}

Presence stanzas are used to inform nodes on the network of the availability of
other nodes.  Presence goes beyond simply online/offline; resources can mark themselves
as away or do not disturb as well.  With certain extensions, presence can also include
what music a user is listening to, where they are, etc.
\footnote{These forms of rich presence are actually sent with message stanzas, but you get the point.}

\pause
\subsection*{Message}

Messages are one-way, fire-and-forget stanzas that transfer data.  The data can be chat messages, notifications,
feed content, or really anything.

\begin{minted}{xml}
<message type='chat' to='another-user@talkr.im' from='hoelzro@talkr.im/Home'>
  <body>Hello, World!</body>
</message>
\end{minted}

\pause
\subsection*{IQ}

IQs are stanzas that transfer data like messages, but they always expect a response.  An example of an IQ would be
sending a configuration change to a chatroom; you send the change, and you want to make sure that the configuration
succeeded or failed.  Another good example would be a ping; each ping should be replied to.
