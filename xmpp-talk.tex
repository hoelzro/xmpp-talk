\documentclass{article}

\usepackage[display]{texpower}
\usepackage[landscape]{geometry}
\usepackage{hyperref}

\pagestyle{empty}
\title{
  \indent An Introduction to XMPP \newline
  Madison Perl Mongers
  \thanks{This presentation was created using \LaTeX.}}
\author{Rob Hoelz}
\date{December 2010}

\begin{document}

\maketitle

\newpage
\part{Introduction}

\newpage
This section will cover the basic idea of what XMPP is.

\newpage
\part{Setup}

\newpage
This section will cover some of the open-source XMPP servers, and
how to set one up.  In this talk, we'll be using Prosody (\url{http://prosody.im}).

\newpage
\part{Concepts}

\newpage
This section will cover some basic concepts in XMPP.  Highly abstract; little to no
actual code.

\begin{itemize}
\item JIDs
\item Domains vs Hosts
\item Types of Stanzas
\end{itemize}

\newpage
\part{Clients}

\newpage
This section covers the various clients available for different operating systems.
\url{http://xmpp.org/xmpp-software/clients}

\newpage
\part{Client Libraries}

\newpage
There are many open-source implementations of XMPP out there; for a comprehensive list,
please consult \url{http://xmpp.org/xmpp-software/libraries}.

\newpage
Since this is a MadMongers meeting, we'll be covering AnyEvent::XMPP.

\newpage

\vspace*{\fill}
\begin{center}
\textbf{\Huge Questions?}
\end{center}
\vspace*{\fill}

\newpage
\appendix
\section{Resources}

\end{document}
