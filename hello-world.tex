\section{Hello, World!}

\pause

Here's one of those boring "Hello, World" programs I was talking about, but XMPP style!

\begin{shaded}
\inputminted{perl}{examples/hello.pl}
\end{shaded}

\newpage
\section{"Hello, World!" Explained}

\pause

\begin{shaded}
\begin{minted}{perl}
use AnyEvent::XMPP::IM::Connection;
\end{minted}
\end{shaded}

\pause
\begin{shaded}
\begin{minted}{perl}
my $cond = AnyEvent->condvar;
\end{minted}
\end{shaded}

\pause
\begin{shaded}
\begin{minted}{perl}
my $conn = AnyEvent::XMPP::IM::Connection->new(
    username => 'bot',
    domain   => 'chat.madmongers',
    resource => scalar(getpwuid $<),
    host     => 'localhost',
    port     => 5222,
    password => 'abc123',
);
\end{minted}
\end{shaded}

\pause
Let's take a break from the code and look at the 'username', 'domain', and
'resource' options more carefully.

\pause
Another way of specifying these three options is with the 'jid' option:

\begin{shaded}
\begin{minted}{perl}
my $conn = AnyEvent::XMPP::IM::Connection->new(
    jid      => 'bot@chat.madmongers/rhoelz',
    host     => 'localhost',
    port     => 5222,
    password => 'abc123',
);
\end{minted}
\end{shaded}

(for comparison):

\begin{shaded}
\begin{minted}{perl}
my $conn = AnyEvent::XMPP::IM::Connection->new(
    username => 'bot',
    domain   => 'chat.madmongers',
    resource => scalar(getpwuid $<),
    host     => 'localhost',
    port     => 5222,
    password => 'abc123',
);
\end{minted}
\end{shaded}

\pause
The 'jid' option allows you to provide a JID.  A JID is short for \textit{Jabber ID}, and it's how
you identify users on a network.

\newpage

\section{JIDs}

\pause

Here's an example of a JID:

\jid{rob@chat.madmongers/Laptop}

\pause

Every JID has three parts:

\pause

\begin{description}
\item[Username] rob
\pause
\item[Domain] chat.madmongers
\pause
\item[Resource] Laptop
\end{description}

\pause

You can leave off the resource part of a JID; that makes what's called a \textit{bare JID}.  A JID with a resource
is known as a \textit{full JID}.  The difference plays a role in how the server moves messages around; more on this later.

\newpage

Back to the code...

\begin{shaded}
\begin{minted}{perl}
$conn->reg_cb(session_ready => sub {
    my $msg = AnyEvent::XMPP::IM::Message->new(
        type => 'chat',
        to   => $to,
        body => 'Hello, World!',
    );

    $msg->send($conn);

    my $timer;
    $timer = AnyEvent->timer(
        after => 3,
        cb    => sub {
            undef $timer;
            $cond->send;
        },
    );
});
\end{minted}
\end{shaded}

\pause
\begin{shaded}
\begin{minted}{perl}
$conn->connect;
\end{minted}
\end{shaded}

\pause
\begin{shaded}
\begin{minted}{perl}
$cond->recv;
\end{minted}
\end{shaded}
