\section{Extensions}

\pause

We've covered the M and the two Ps, but what about the X?

\pause

The 'X' in XMPP stands for e\textbf{X}tensible!  Since stanzas are simply chunks of XML,
we can throw other XML chunks inside of them (you saw this with my IQ ping example).

\pause
You've already seen the MUC and Ping extensions in action, here's some other extensions that
are available:

\pause
\begin{itemize}
\item Disco
\pause
\item PubSub
\pause
\item BOSH
\pause
\item XHTML-IM
\pause
\item vCard
\pause
\item Ad-hoc Commands
\pause
\item SOAP over XMPP
\pause
\item File Transfer
\pause
\item Jingle
\pause
\item Data Forms
\pause
\item In-Band Registration
\end{itemize}

\pause

Note that these extensions are merely specifications; a server or client can choose whether or to implement
them.  For example, Gajim can use ad-hoc commands, while Pidgin cannot; and ejabberd implements PubSub, while
Prosody does not.  In addition, a server can choose to implement additional extensions that aren't in the list
of officially published list of extensions; Prosody implements an extension that automatically truncates messages
that are too long and replaces them with a link to a pastebin.
