\section{Servers}

\pause

Since XMPP is a decentralized protocol, you'll need a server before you can get started using it.
If you're just interested in chatting, you may want to set up an account at a public XMPP service,
such as Google Talk, or \url{talkr.im}.

\pause
An alternative to registering at a public XMPP service is to host your own.  This is useful if you want
full control of your data or your server, or if you want your server to be internally visible only.  Also,
some public servers' configurations may not fulfill your needs, or it might have restrictions imposed on
connecting clients to prevent them from overloading the server.

\pause
There are several implementations of XMPP servers out there; in this presentation, I'll be using Prosody,
which is written in Lua and a personal favorite of mine.  Two other servers of note are djabberd (noteworthy
because it is written in Perl, but it doesn't look very active) and ejabberd (which is written in Erlang, but is
noteworthy because of its popularity and wide availability of extensions).

\pause
I have a Prosody server running on my laptop right now; if you'd like to follow along, feel free to connect to my ad-hoc
network and create an account.  The domain name for the server is \textbf{chat.madmongers}.  The configuration syntax for Prosody is
fairly simple (it's just Lua), and you don't usually need to deviate much from the configuration provided in the distribution to
get up and running, but for the sake of brevity, I've omitted the configuration file from the presentation itself.  If you'd
like to take a look at the configuration I'm using, you can take a look here:

\pause
\example{prosody.cfg.lua}
